%
% Portuguese-BR vertion
% 
\documentclass{report}

\usepackage{ipprocess}
\usepackage[utf8]{inputenc} 
\usepackage[brazil]{babel} % Uncomment for portuguese
\usepackage{longtable}
\usepackage{multirow}
\usepackage{graphicx}

\sloppy

\graphicspath{{./pictures/}} % Pictures dir
\makeindex
\begin{document}

\DocumentTitle{Documento de Arquitetura}
\Project{ANTARES-R2}
\Organization{FAZEMOS QUALQUER NEGÓCIO INC.}
\Version{Build 3.0a}

\capa
\newpage
\newpage

%%%%%%%%%%%%%%%%%%%%%%%%%%%%%%%%%%%%%%%%%%%%%%%%%%
%% Revision History
%%%%%%%%%%%%%%%%%%%%%%%%%%%%%%%%%%%%%%%%%%%%%%%%%%
\chapter*{Histórico de Revisões}
  \vspace*{1cm}
  \begin{table}[ht]
    \centering
    \begin{tabular}[pos]{|m{2cm} | m{8cm} | m{4cm}|} 
      \hline
      \cellcolor[gray]{0.9}
      \textbf{Date} & \cellcolor[gray]{0.9}\textbf{Descrição} & \cellcolor[gray]{0.9}\textbf{Autor(s)}\\
      \hline
      12/08/2016 &  Concepção do documento & Manuelle Macedo \\ \hline    
      13/08/2016 &  Codificação das instruções e Estruturação do documento & Manuelle Macedo \\ \hline
      14/08/2016 &  Requisitos funcionais e não funcionais & Manuelle Macedo \\ \hline
      15/08/2016 &  Finalização da codificação das instruções e revisão geral & Manuelle Macedo \\ \hline
      14/08/2016 &  Descrição sobre o Assembler e características gerais & João Pedro \\ \hline
    \end{tabular}
  \end{table}

% TOC instantiation
\tableofcontents

%%%%%%%%%%%%%%%%%%%%%%%%%%%%%%%%%%%%%%%%%%%%%%%%%%
%% Document main content
%%%%%%%%%%%%%%%%%%%%%%%%%%%%%%%%%%%%%%%%%%%%%%%%%%
\chapter{Introdução}
  
  \section{Propósito do Documento}
  Este documento descreve parte da arquitetura do projeto \ipPROCESSProject, incluindo informações sobre as instruções, a memória compartilhada, e alguns elementos do processador, bem como o funcionamento do assembler para tal. Ele também apresenta informações sobre o assembler construído para que qualquer código em Assembly respeitando nossas instruções, rode no IP-Core. O principal objetivo deste documento é definir as especificações do projeto \ipPROCESSProject e prover uma visão geral completa do mesmo.
  
  \section{Stakeholders}
    \FloatBarrier
    \begin{table}[H] 
      \begin{center}
        \begin{tabular}[pos]{|m{6cm} | m{8cm}|} 
          \hline 
          \cellcolor[gray]{0.9}\textbf{Nome} & \cellcolor[gray]{0.9}\textbf{Papel/Responsabilidades} \\ \hline 
          Manuelle	& Análise \\  \hline 
          João Pedro	& Projeto \\  \hline 
        \end{tabular}
      \end{center}
    \end{table} 

\section{Visão Geral do Documento}

O presente documento é apresentado como segue:

  \begin{itemize}
   \item \textbf{Capítulo 2 --} Este capítulo apresenta uma visão geral da arquitetura, falando um pouco sobre as instruções e algumas características gerais do processador, bem como o assembler.
  \end{itemize}


%   % inicio das definições do documento
%   \section{Definições}
%     \FloatBarrier
%     \begin{table}[H]
%       \begin{center}
%         \begin{tabular}[pos]{|m{5cm} | m{9cm}|} 
%           \hline
%           \cellcolor[gray]{0.9}\textbf{Termo} & \cellcolor[gray]{0.9}\textbf{Descrição} \\ \hline
%                           &                       \\ \hline
%         \end{tabular}
%       \end{center}
%     \end{table}  
%   % fim

  % inicio da tabela de acronimos e abreviacoes do documento
  \section{Acrônimos e Abreviações}
    \FloatBarrier
    \begin{table}[H]
      \begin{center}
        \begin{tabular}[pos]{|m{2cm} | m{12cm}|} 
          \hline
          \cellcolor[gray]{0.9}\textbf{Sigla} & \cellcolor[gray]{0.9}\textbf{Descrição} \\ \hline
              OPCODE  & Código de operação  \\ \hline
              PC  & Contador de Programa  \\ \hline
              RD  & Registrador de destino  \\ \hline
              RS, RT  & Registrador de origem dos operandos\\ \hline
              HI  & 32 bits mais significativos do acumulador  \\ \hline
              LO  & 32 bits menos significativos do acumulador  \\ \hline
        \end{tabular}
      \end{center}
    \end{table}  
  % fim

\chapter{Visão Geral da Arquitetura}

    \section{Características Gerais}
A arquitetura a qual foi proposta é dotada das seguintes características:  32 Bits para instruções, 32 registradores de propósito geral, o nível ISA composto por 64 instruções sendo que  6 delas são pseudo-Instruções, a organização da memória é dada de forma Big Endian, Tendo essa memoria como compartilhada  e tendo dois registradores especiais para operações com o acumulador, além de trabalhar com 5 modos de endereçamento.
  \section{Codificação das instruções}
  Nesta arquitetura temos 64 instruções distribuídas da seguinte forma: 40 instruções de operações aritméticas, lógicas, multiplicação/divisão e de deslocamento, 6 de testes condicionais e operações condicionais de troca, 4 de operações de acesso ao acumulador, 8 de saltos e \textit{branches} e por fim mais 6 de operações de \textit{load} e \textit{store}. Todas as instruções contém 32 bits. Exitem três formatos de instruções: \textit{R} (\textit{R-type}), \textit{I} (\textit{I-type}) e \textit{Jump}. Ainda nesta sessão os OPCODES de cada instrução serão apresentados de forma binaria.
  
  \FloatBarrier
    \begin{center}
\begin{longtable}[pos]{|m{2cm} | m{1cm} | m{8cm}|} \hline    
          \multicolumn{1}{|c|}{\cellcolor[gray]{0.9}\textbf{Formato da instrução}} & 
          \multicolumn{1}{c|}{\cellcolor[gray]{0.9}\textbf{Instrução}} & 
          \multicolumn{1}{l|}{\cellcolor[gray]{0.9}\textbf{Descrição}} \\ \hline
          \endfirsthead
          \hline
          \multicolumn{3}{|l|}
          {{\bfseries continuação da página anterior}} \\
          \hline
          \multicolumn{1}{|c|}{\cellcolor[gray]{0.9}\textbf{Formato da Instrução}} & 
          \multicolumn{1}{c|}{\cellcolor[gray]{0.9}\textbf{Instrução}} & 
          \multicolumn{1}{l|}{\cellcolor[gray]{0.9}\textbf{Descrição}} \\ \hline
          \endhead
            \hline
          \multicolumn{3}{|r|}{{\bfseries continua na próxima página}} \\ \hline
          \endfoot

          \hline
          \endlastfoot 
           
			\multirow{30}{*}{R-type} & ADD & Soma dois valores \\ \cline{2-3}
			& ADDU & Soma dois valores sem sinal \\ \cline{2-3}
			& CLO & Conta quantos 1's existem no registrador $RS$ da esquerda para a direita até o primeiro 0 \\ \cline{2-3}
			& CLZ &  Conta quantos 0's existem em $RS$ da esquerda para a direita até o primeiro 1\\ \cline{2-3}
			& SUB & Subtrai dois valores  \\ \cline{2-3}
			& SUBU & Subtrai dois valores sem sinal \\ \cline{2-3}
			& SEB & Estende o dado em 1 byte \\ \cline{2-3}
			& SEH & Estende o dado em metade de uma palavra, ou seja,  2 bytes \\ \cline{2-3}
			& SLL & Desloca a palavra para a esquerda em 5 bits \\ \cline{2-3}
			& SLLV & Desloca $RS$ para a esquerda pelo valor dos 5 bits menos significativos de$RT$ \\ \cline{2-3}
			& SRA & Desloca a palavra para a direita em 5 bits sinalizados \\ \cline{2-3}
			& SRAV & Desloca $RS$ para a direita pelo valor dos 5 bits menos significativos de $RT$ \\ \cline{2-3}
			& SRL &  Desloca a palavra para a direita em 5 bits \\ \cline{2-3}
			& SRLV & Desloca $RS$ para a direita pelo valor dos 5 bits, sem sinal, menos significativos de $RT$  \\ \cline{2-3}
			& AND & AND lógico de dois valores \\ \cline{2-3}
			& NOR & OR lógico de dois valores negado \\ \cline{2-3}
			& OR &  OR lógico de dois valores\\ \cline{2-3}
			& XOR & OR lógico exclusivo de dois valores \\ \cline{2-3}
			& DIV &  Divisão de dois valores onde o resto fica no registrador $HI$ e o quociente no $LO$\\ \cline{2-3}
			& DIVU &  Divisão de dois valores sem sinal onde o resto fica no registrador $HI$ e o quociente no $LO$  \\ \cline{2-3}
			& MADD &  Multiplica dois valores e soma a outro valor anterior\\ \cline{2-3}
			& MADDU & Multiplica dois valores sem sinal e soma a outro valor anterior \\ \cline{2-3}
			& MSUB &  Multiplica dois valores e subtrai a outro valor anterior\\ \cline{2-3}
			& MSUBU & Multiplica dois valores sem sinal e subtrai a outro valor anterior \\ \cline{2-3}
			& MUL & Multiplica dois valores  \\ \cline{2-3}
			& MULT & Multiplica dois valores de 32 bits e salva em dois registradores $HI$ e $LO$ \\ \cline{2-3}
			& MULTU & Multiplica dois valores sem sinal de 32 bits e salva em dois registradores $HI$ e $LO$ \\ \cline{2-3}
			& MFHI & Move um dado de $RD$ para $HI$ \\ \cline{2-3}
			& MFLO & Move um dado de $RD$ para $LO$ \\ \cline{2-3}
			& MTHI & Move um dado de $HI$ para $RS$ \\ \cline{2-3}
			& MTLO & Move um dado de $LO$ para $RS$ \\ \cline{2-3}
			& JR & Pula para o endereço que está em $RS$ \\ \cline{2-3}
			& JALR & Guarda o valor atual de PC mais 4 em $RD$ e pula para o endereço que está em $RS$ \\ \cline{2-3}
			& MOVN & Move o que está em $RS$ para $RD$ se o dado de $RT$ for diferente de 0  \\ \cline{2-3}
			& MOVZ & Move o que está em $RS$ para $RD$ se o dado de $RT$ for igual a 0  \\ \cline{2-3}
			& SLT & Guarda em $RD$ 1, se o dado de $RS$ for menor que o de $RT$, senão guarda 0 \\ \cline{2-3}
			& SLTU & Guarda em $RD$ 1, se o dado de $RS$ for menor que o de $RT$, sendo ambos sem sinal, senão guarda 0 \\ \cline{2-3}
			& WSBH &  \\ \cline{2-3}
			& MOVE & Atribui o valor do registrador $RS$ ao $RD$ \\\hline
			& NOP & Não realiza nenhuma operação \\\hline
	\multirow{20}{*}{I-type} & ADDI & Soma dois valores, um destes imediato \\ \cline{2-3}
	& ADDIU & Soma dois valores, um destes imeditao, ambos sem sinal \\ \cline{2-3}
	& LUI & Desloca para a direita o imediato, em 16 bits \\ \cline{2-3}
	& ANDI & AND lógico de dois valores, um destes imediato \\ \cline{2-3}
	& ORI & OR lógico de dois valores, um destes imediato \\ \cline{2-3}
	& XORI & OR exclusivo de dois valores, um destes imediato \\ \cline{2-3}
	& BGEZ & Altera o valor de PC para o PC atual mais o imediato de 18 bits, se o dado em $RS$ for maior igual a 0 \\ \cline{2-3}
	& BLTZ & Altera o valor de PC para o PC atual mais o imediato de 18 bits, se o dado em $RS$ for menor que 0 \\ \cline{2-3}
	& BEQ & Altera o valor de PC para o PC atual mais o imediato de 18 bits, se o dado em $RS$ for igual ao de $RT$ \\ \cline{2-3}
	& BNE &  Altera o valor de PC para o PC atual mais o imediato de 18 bits, se o dado em $RS$ for diferente ao de $RT$\\ \cline{2-3}
	& LB & Busca na posição $RS+Offset16$ e guarda em $RT$ o byte \\ \cline{2-3}
	& LW & Busca na posição $RS+Offset16$ e guarda em $RT$ a palavra \\ \cline{2-3}
	& SB & Guarda o byte que está em $RT$, na posição $RS+Offset16$ da memória \\ \cline{2-3}
	& SW & Guarda a palavra que está em $RT$, na posição $RS+Offset16$ da memória \\ \cline{2-3}
	& LH & Busca na posição $RS+Offset16$ e guarda em $RT$ um dado de 16 bits\\ \cline{2-3}
	& SH & Guarda um dado de 16 bits que está em $RT$, na posição $RS+Offset16$ da memória \\ \cline{2-3}
	& SLTI & Guarda em $RD$ 1, se o dado de $RS$ for menor que o valor imediato, senão guarda 0 \\ \cline{2-3}
	& SLTIU & Guarda em $RD$ 1, se o dado de $RS$ sem sinal for menor que o valor imediato, senão guarda 0 \\ \cline{2-3}
	& LA & Carrega o valor de uma label como imediato \\ \cline{2-3}
	& LI & Carrega uma constantede 32 bits ao registrador $RD$ \\\hline
	\multirow{2}{*}{Jump} & J & Altera o PC para o endereço vindo na instrução \\ \cline{2-3}
	& JAL & Guarda o PC mais 4 em \textit{RA} e altera o PC para o endereço vindo na instrução \\\hline
\end{longtable}
\end{center}

	O formato \textit{R} está relacionado as instruções lógicas, aritméticas, deslocamento, acesso ao acumulador e operações condicionais.
	\begin{figure}[H]
    	\centering
    	\includegraphics{R-type}
	\end{figure}
	
\begin{table}[H]
\centering	
\begin{tabular}{|c|l|}
	\hline 
	\cellcolor[gray]{0.9}\textbf{CAMPO} & \cellcolor[gray]{0.9}\textbf{DESCRIÇÃO} \\ 
	\hline 
	\textit{OPCODE} & Código da operação básica da instrução. \\ 
	\hline 
	\textit{RS} & Registrador do primeiro operando de origem. \\ 
	\hline 
	\textit{RT} & Registrador do segundo operando de origem. \\ 
	\hline 
	\textit{RD} & Registrador de destino. \\ 
	\hline 
	\textit{SHAMT} & Quantidade de deslocamento. \\ 
	\hline 
	\textit{FUNCT} & Variante específica da operação. \\ 
	\hline 
	\end{tabular} 
	\end{table}
	
	Um segundo tipo de formato de instrução é chamado de formato \textit{I}, utilizado pelas instruções imediatas, lógicas.
	\begin{figure}[H]
    	\centering
    	\includegraphics{I-Type}
  	\end{figure}
  	
  	\begin{table}[H]
\centering	
\begin{tabular}{|c|l|}
	\hline 
	\cellcolor[gray]{0.9}\textbf{CAMPO} & \cellcolor[gray]{0.9}\textbf{DESCRIÇÃO} \\ 
	\hline 
	\textit{OPCODE} & Código da operação básica da instrução. \\ 
	\hline 
	\textit{RS} & Registrador do primeiro operando de origem. \\ 
	\hline 
	\textit{RT} & Registrador de destino. \\ 
	\hline 
	\textit{IMMEDIATE} & Constante numérica. \\ 
	\hline 
	\end{tabular} 
	\end{table}
	
	O formato \textit{Jump} serve para as instruções de desvio condicional e incondicional. 
 	
   	\begin{figure}[H]
    	\centering
    	\includegraphics{J-Type}
		\label{jump}
  	\end{figure}
  	
  	\begin{table}[H]
\centering	
\begin{tabular}{|c|l|}
	\hline 
	\cellcolor[gray]{0.9}\textbf{CAMPO} & \cellcolor[gray]{0.9}\textbf{DESCRIÇÃO} \\ 
	\hline 
	\textit{OPCODE} & Código da operação básica da instrução. \\ 
	\hline 
	\textit{ADDRESS} & Endereço de memória. \\ 
	\hline 
	\end{tabular} 
	\end{table}

Instruções Aritméticas
\begin{table}[H]
\centering
	\begin{tabular}{|c|c|c|}
  	\hline 
  	\cellcolor[gray]{0.9}\textbf{OPCODE} & \cellcolor[gray]{0.9}\textbf{INSTRUCTION} & \cellcolor[gray]{0.9}\textbf{FUNCTION} \\ 
  	\hline 
  	000000 & ADD & 100000 \\ 
  	\hline 
  	001000 & ADDI & -\\ 
  	\hline 
  	001001 & ADDIU & - \\ 
  	\hline 
  	000000 & ADDU & 100001 \\ 
  	\hline 
  	000000 & CLO & 010001 \\ 
  	\hline 
  	000000 & CLZ & 100000 \\ 
  	\hline  
  	000000 & SUB & 100010 \\ 
  	\hline
  	000000 & SUBU & 100011 \\
  	\hline 
  	011111 & SEB & 100000 \\
  	\hline 
  	011111 & SEH & 100000 \\
  	\hline 
  	001111 & LUI & - \\
  	\hline 
  	001001 & LA & - \\
  	\hline
  	001101 & LI & - \\
  	\hline
  	000000 & MOVE & 100000 \\
  	\hline
  	
  	\end{tabular} 
  \end{table} 
  	 	
Operações de Multiplicação e Divisão
\begin{table}[H]
\centering
	\begin{tabular}{|c|c|c|}
  	\hline 
  	\cellcolor[gray]{0.9}\textbf{OPCODE} & \cellcolor[gray]{0.9}\textbf{INSTRUCTION} & \cellcolor[gray]{0.9}\textbf{FUNCTION} \\ 
  	\hline 
  	000000 & DIV & 011010 \\ 
  	\hline 
  	000000 & DIVU & 011011 \\ 
  	\hline 
  	011100 & MADD & 000000 \\ 
  	\hline 
  	011100 & MADDU & 000001 \\ 
  	\hline  
  	011100 & MSUB & 000100 \\ 
  	\hline
  	011100 & MSUBU & 000101 \\
  	\hline 
  	011100 & MUL & 000010 \\
  	\hline 
  	000000 & MULT & 011000 \\
  	\hline
  	000000 & MULTU & 011001 \\
  	\hline
  	\end{tabular} 
  \end{table} 
	
Instruções Lógicas
\begin{table}[H]
\centering	
\begin{tabular}{|c|c|c|}
	\hline 
  \cellcolor[gray]{0.9}\textbf{OPCODE} & \cellcolor[gray]{0.9}\textbf{INSTRUCTION} & \cellcolor[gray]{0.9}\textbf{FUNCTION} \\ 
  	\hline 
	000000 & AND & 100100 \\ 
	\hline 
	001100 & ANDI & -\\ 
	\hline 
	000000 & NOR & 100111\\ 
	\hline 
	000000 & OR & 100101\\ 
	\hline
	001101 & ORI & -\\ 
	\hline 
	000000 & XOR & 100110 \\
	\hline
	001110 & XORI & -\\ 
  	\hline 
  	000000 & NOP & 000000\\ 
  	\hline
	\end{tabular} 
\end{table}	
	
Instruções de Acesso ao Acumulador
\begin{table}[H]
\centering 	
  	\begin{tabular}{|c|c|c|}
  	\hline 
  	\cellcolor[gray]{0.9}\textbf{OPCODE} & \cellcolor[gray]{0.9}\textbf{INSTRUCTION} & \cellcolor[gray]{0.9}\textbf{FUNCTION} \\ 
  	\hline 
  	000000 & MFHI & 010000 \\ 
  	\hline 
  	000000 & MFLO & 010010 \\ 
  	\hline 
  	000000 & MTHI & 010001 \\ 
  	\hline 
  	000000 & MTLO & 010011 \\ 
  	\hline 
  	\end{tabular} 
\end{table}

Instruções de Deslocamento
\begin{table}[H]
\centering 	
  	\begin{tabular}{|c|c|c|}
  	\hline 
  	\cellcolor[gray]{0.9}\textbf{OPCODE} & \cellcolor[gray]{0.9}\textbf{INSTRUCTION} & \cellcolor[gray]{0.9}\textbf{FUNCTION} \\ 
  	\hline 
  	000000 & SLL & 000000\\ 
  	\hline 
  	000000 & SLLV & 000100\\ 
  	\hline 
  	000000 & SRA & 000011 \\ 
  	\hline 
  	000000 & SRAV & 000111 \\ 
  	\hline
  	000000 & SRL & 000010 \\ 
  	\hline 
  	000000 & SRLV & 000110 \\ 
  	\hline 
  	\end{tabular} 
\end{table}

Instruções Condicionais
\begin{table}[H]
\centering 	
  	\begin{tabular}{|c|c|c|}
  	\hline 
  \cellcolor[gray]{0.9}\textbf{OPCODE} & \cellcolor[gray]{0.9}\textbf{INSTRUCTION} & \cellcolor[gray]{0.9}\textbf{FUNCTION} \\ 
  	\hline  
  	000000 & MOVN & 001011 \\ 
  	\hline 
  	000000 & MOVZ & 001010 \\ 
  	\hline 
  	000000 & SLT & 101010 \\ 
  	\hline 
  	001010 & SLTI & -\\ 
  	\hline 
  	001011 & SLTIU & -\\ 
  	\hline 
  	000000 & SLTU & 101011 \\ 
  	\hline 
  	\end{tabular} 
\end{table}

Instruções de Jump e Branch
\begin{table}[H]
\centering 	
  	\begin{tabular}{|c|c|c|}
  	\hline 
  	 \cellcolor[gray]{0.9}\textbf{OPCODE} & \cellcolor[gray]{0.9}\textbf{INSTRUCTION} & \cellcolor[gray]{0.9}\textbf{FUNCTION} \\ 
  	\hline  
  	000001 & BQEZ &-\\ 
  	\hline 
  	000001 & BLTZ &-\\ 
  	\hline 
  	000100 & BEQ &-\\ 
  	\hline 
  	000101 & BNE &-\\ 
  	\hline 
  	000010 & J &-\\ 
  	\hline 
  	000000 & JR & 001000 \\ 
  	\hline
  	000011 & JAL &-\\ 
  	\hline
  	000000 & JALR & 001001 \\ 
  	\hline
  	
  	\end{tabular} 
\end{table}

Instruções de Load e Store
\begin{table}[H]
\centering 	
  	\begin{tabular}{|c|c|}
  	\hline 
  	\cellcolor[gray]{0.9}\textbf{OPCODE} & \cellcolor[gray]{0.9}\textbf{INSTRUCTION} \\ 
  	\hline 
  	100000 & LB \\ 
  	\hline 
  	100011 & LW \\ 
  	\hline 
  	101000 & SB \\ 
  	\hline 
  	101011 & SW \\ 
  	\hline 
  	100001 & LH \\ 
  	\hline
  	101001 & SH \\ 
  	\hline 
  	\end{tabular} 
\end{table}
  
  \section{Requisitos}
  
  \subsection{Funcionais}
  Todas as instruções são requisitos essenciais pois sem elas o módulo não funciona como deveria. Por ser uma arquitetura de 32 bits, as palavras de entrada e saída tem 32 bits. O módulo ainda conta com \textit{flags} de zero, \textit{overflow}, \textit{carry} e negativo.
  
  \subsection{Não Funcionais}
  A memória possui o tamanho de 64KB e é compartilhada, ou seja, é a mesma memória para as instruções e dados. A palavra tem o tamanho de 32 bits, ou seja os dados tem o mesmo tamanho que uma instrução. Os dados são colocados na memória usando o formato Big Endian, onde os bit mais significativos vem primeiro, na posição de memória.  Temos um banco de registradores 32 registradores de proposito geral e mais 2 para acumulador chamados de LI e HO. Na tabela abaixo tem-se a descrição dos 32 registradores.
  
  \FloatBarrier
    \begin{table}[H]
      \begin{center}
        \begin{tabular}[pos]{|m{3cm} | m{3cm}| m{8cm} |} 
          \hline
          \cellcolor[gray]{0.9}\textbf{Número do Registrador} & \cellcolor[gray]{0.9}\textbf{Nome Alternativo} & \cellcolor[gray]{0.9}\textbf{Descrição} \\ \hline
              0  & zero & O valor 0 \\ \hline
              1  & \$at & Reservado para o Assembler\\ \hline
              2-3  & \$v0-\$v1 & Valores de expressões e resultados de funções\\ \hline
              4-7  & \$a0-\$a3 & Primeiros 4 parâmetros de sub-rotinas.\\ \hline
              8-15  & \$t0-\$t7 & Chamadas  \\ \hline
              16-23  & \$s0-\$s7 &  Variáveis Salvas\\ \hline
              24-25  & \$t8-\$t9 & Temporários \\ \hline
              26-27  & \$k0-\$k1 &  OS temporário \\ \hline
              28  & \$gp & Global Point \\  \hline
              29  & \$sp & Stack point \\ \hline
              30  & \$s8/\$fp & Frame point\\ \hline
              31  & \$ra & Procedimento de Return de Endereço \\ \hline
        \end{tabular}
      \end{center}
    \end{table}  
 
  
  \section{Assembler}
  
  O Assembler esta implementado da seguinte forma, para realizar a montagem da instrução primeiramente é preciso verificar todas as linhas, nesta verificação, são retirados os comentários, e para que possamos medir o tamanho do programa que será gerado. Esta informação é relevante para que ao terminar a montagem do programa, possamos reservar um espaço para a memoria de dados, e realizando a contagem, podemos definir a partir de qual endereço os dados serão gravados. \\
    Outra informação importante nesta verificação é a identificação das \textit{labels}, neste momento, nós procuramos todas as \textit{labels}, e guardados a linha na qual ela está sendo chamada. Com essa informação guardada, toda vez que for chamada essa \textit{label} em um \textit{branch} ou \textit{jump}, nós traduziremos para o endereço na qual esta \textit{label} está sendo referenciada.
    Outra parte essencial do Assembler é o verificador de instruções, nele após esse estagio de pré processamento, nós identificamo as instruções do tipo I, J e R ,Não foram implementados os \textit{branchs}, ao recebermos a instrução, primeiramente nós identificamos qual o tipo dela, ou seja se ela faz de parte de algum dos grupos de instruções que temos, ao identificar qual é a instrução, sabemos todas as informações que são necessárias para podermos traduzir para o seu binário correspondente, sabemos o seu opcode, quantos registradores virão junto a instrução, e onde esses dados devem ser colocados em cada seguimento do binário de saída, por exemplo, sabemos que os 6 primeiros opcodes, e que os 11 últimos bits de instruções do tipo R correspondem a instruções ao \textit{Shift} e o \textit{Funct} da instrução.
    
    Para executar o Assembler é preciso a linguagem \textit{python} instalada, e rodar a classe \textit{main.py}, dentro dela é possível modificar qual arquivo será lido para tradução.
    
  

% Optional bibliography section
% To use bibliograpy, first provide the ipprocess.bib file on the root folder.
% \bibliographystyle{ieeetr}
% \bibliography{ipprocess}

\end{document}
